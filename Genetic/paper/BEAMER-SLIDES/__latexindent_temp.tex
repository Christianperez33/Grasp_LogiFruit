\documentclass[spanish]{beamer}

\usepackage[spanish]{babel}
\selectlanguage{spanish}
\usepackage[utf8]{inputenc}

\mode<presentation> {

% The Beamer class comes with a number of default slide themes
% which change the colors and layouts of slides. Below this is a list
% of all the themes, uncomment each in turn to see what they look like.

\usetheme{default}
%\usetheme{AnnArbor}
%\usetheme{Antibes}
%\usetheme{Bergen}
%\usetheme{Berkeley}
%\usetheme{Berlin}
%\usetheme{Boadilla}
%\usetheme{CambridgeUS}
%\usetheme{Copenhagen}
%\usetheme{Darmstadt}
%\usetheme{Dresden}
%\usetheme{Frankfurt}
%\usetheme{Goettingen}
%\usetheme{Hannover}
%\usetheme{Ilmenau}
%\usetheme{JuanLesPins}
%\usetheme{Luebeck}
%\usetheme{Madrid}
%\usetheme{Malmoe}
%\usetheme{Marburg}
%\usetheme{Montpellier}
%\usetheme{PaloAlto}
%\usetheme{Pittsburgh}
%\usetheme{Rochester}
%\usetheme{Singapore}
%\usetheme{Szeged}
%\usetheme{Warsaw}

% As well as themes, the Beamer class has a number of color themes
% for any slide theme. Uncomment each of these in turn to see how it
% changes the colors of your current slide theme.

%\usecolortheme{albatross}
%\usecolortheme{beaver}
%\usecolortheme{beetle}
%\usecolortheme{crane}
%\usecolortheme{dolphin}
%\usecolortheme{dove}
%\usecolortheme{fly}
%\usecolortheme{lily}
%\usecolortheme{orchid}
%\usecolortheme{rose}
%\usecolortheme{seagull}
%\usecolortheme{seahorse}
%\usecolortheme{whale}
%\usecolortheme{wolverine}

%\setbeamertemplate{footline} % To remove the footer line in all slides uncomment this line
%\setbeamertemplate{footline}[page number] % To replace the footer line in all slides with a simple slide count uncomment this line

%\setbeamertemplate{navigation symbols}{} % To remove the navigation symbols from the bottom of all slides uncomment this line
}

\usepackage{graphicx} % Allows including images
\usepackage{booktabs} % Allows the use of \toprule, \midrule and \bottomrule in tables

%----------------------------------------------------------------------------------------
%	TITLE PAGE
%----------------------------------------------------------------------------------------

\title{Text Line Extraction Based on Distance Map Features and Dynamic Programming} % The short title appears at the bottom of every slide, the full title is only on the title page

\author{Christian Pérez Bernal} % Your name
\institute[UPV] % Your institution as it will appear on the bottom of every slide, may be shorthand to save space
{
Universidad Politécnica de Valencia \\ % Your institution for the title page
\medskip
\textit{cripeber@upv.es} % Your email address
}
\date{\today} %Date, can be changed to a custom date

\begin{document}

\begin{frame}
\titlepage % Print the title page as the first slide
\end{frame}

\begin{frame}
\frametitle{Overview} % Table of contents slide, comment this block out to remove it
\tableofcontents % Throughout your presentation, if you choose to use \section{} and \subsection{} commands, these will automatically be printed on this slide as an overview of your presentation
\end{frame}

%----------------------------------------------------------------------------------------
%	PRESENTATION SLIDES
%----------------------------------------------------------------------------------------

%------------------------------------------------
\section{Introducción}
%------------------------------------------------

\begin{frame}
\frametitle{Problemática}

\begin{column}{0.5\textwidth}
    \begin{figure}[H]
        \centering
        \includegraphics[width=0.75\textwidth]{hattem}
        \caption{Imagen del corpus Hattem}
    \end{figure}
\end{column}
\begin{column}{0.5\textwidth}
	\begin{itemize}
    	\item[] Segmentación de lineas.
                \vspace{5}
    	\item[] Obtención de la \textit{baseline}.
    	        \vspace{5}
    	\item[] Se emplea un mapa de distancias y programación dinámica.
    	        \vspace{5}

    \end{itemize}
\end{column}

\end{frame}


%------------------------------------------------
\section{Método Propuesto}
\subsection{Idea}
\subsection{Modelo de distancia}
\subsection{Cálculo de frontera}
%------------------------------------------------

\begin{frame}
\frametitle{Mapa de distancias}
\begin{itemize}
    \centering
	\item[] $\gamma \in \psi(x,y), \varrho_{x}(a_{i}) \in [0-255]$
            \vspace{7}
	\item[] $ d_{x}(a_{i},a_{i+1})= \delta_{x}(a_{i},a_{i+1}) + \frac{\varrho_{x}(a_{i+1})}{255} $
	        \vspace{7}
	\item[] $ \Lambda(\gamma )=\sum d_{x}(a_{i},a_{i+1}) $
	        \vspace{7}

\end{itemize}
\begin{figure}[H]
    \centering
    \includegraphics[width=0.75\textwidth]{distmap}
    \caption{Ejemplo de un mapa de distancias}
\end{figure}

\end{frame}

%------------------------------------------------

\begin{frame}
\frametitle{Calculo de frontera}

\begin{itemize}
    \centering
	\item[] $\Theta  \in \Gamma (A_{X}(l_i, l_{i+1})) $
            \vspace{7}
	\item[] $ (a_i,a_{i+1}) \in \Theta$
	        \vspace{7}
	\item[] $P_x (a_i,a_{i+1}) = \delta (a_i,a_{i+1}) + \frac{Max(F_x[A_x(l_i,l_{i+1})]) - F_x(a_{i+1})}{Max(F_x[A_x(l_i,l_{i+1})])} $
	        \vspace{7}
\end{itemize}
\begin{figure}[H]
    \centering
    \includegraphics[width=0.75\textwidth]{frontier}
    \caption{Ejemplo calculo de frontera}
\end{figure}
\end{frame}


%------------------------------------------------
\section{Resultados}
\subsection{Corpus}
\subsection{Medidas de evaluación}
\subsection{Conclusión y resultados}
%------------------------------------------------

\begin{frame}
\frametitle{Corpus}
\begin{block}{ICDAR}
\begin{itemize}
\item[] Conjunto de 150 imágenes con mas de 2600 lineas.
\item[] Imágenes de libros manuscritos.
\item[] Libro en 3 idiomas (Inglés, Griego, Bengalí)
\item[] Labelización por píxel.
\end{itemize}
\end{block}

\begin{block}{Hattem}
\begin{itemize}
\item[] Conjunto de 572 imágenes del siglo XV.
\item[] Imágenes de libros manuscritos en Alemán.
\item[] Corpus final de 40 imágenes seleccionadas por expertos.
\item[] Gran cantidad de abreviaturas.
\end{itemize}
\end{block}
\end{frame}

%------------------------------------------------

\begin{frame}
\frametitle{Medidas de avaluación}
\begin{columns}[c] % The "c" option specifies centered vertical alignment while the "t" option is used for top vertical alignment

\column{.45\textwidth} % Left column and width
\begin{enumerate}
\item \textit{Match Score}
\item \textit{Detection Rate}
\item \textit{Region Accuracy}
\item \textit{F-Measure}
\end{enumerate}

\column{.5\textwidth} % Right column and width
\begin{itemize}
\item $ M_s(i,j) = \frac{\mid G_j \cap Ri \cap I \mid }{\mid (G_j \cup  Ri) \cap I \mid }$
\item $D_r = \frac{o2o}{N}$
\item $R_a = \frac{O2}{M}$
\item $F_m = \frac{2D_r + Ra}{D_r + R_a}$
\end{itemize}

\end{columns}
\end{frame}

%------------------------------------------------

\begin{frame}
\frametitle{Resultados}
\begin{figure}[H]
    \centering
    \includegraphics[width=1\textwidth]{result.png}
    \caption{Comparación de resultados con las diferentes medidas}
\end{figure}
\end{frame}


%----------------------------------------------------------------------------------------

\end{document}